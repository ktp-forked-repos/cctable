\def\TheAcknowledgments{The author would like to thank Jan Wielemaker, the
creator of SWI Prolog, for being very responsive in making enhancements to
and correcting problems in the implementation of delimited continuations 
in SWI Prolog, as well as being very helpful in explaining some of the
internal details of the implementation.}

\def\TheAuthors{Samer Abdallah (\texttt{samer@jukedeck.com})}
\def\TheInstitution{Jukedeck Ltd.}

\def\TheTitle{More declarative tabling in Prolog using multi-prompt delimited control}

\def\TheAbstract{%
Several Prologs include an implementation of \emph{tabling}, an alternative
resolution strategy which uses memoisation to avoid redundant duplication of
computations. Until relatively recently, tabling has required low-level
support in the underlying Prolog engine. An alternative approach is to 
augment Prolog with low level support for continuation capturing control operators,
particularly \emph{delimited continuations}, which have been investigated in the 
field of function programming and found to be capable of supporting a wide variety 
of computational effects within an otherwise declarative language. 

This technical report describes an implementation of tabling in SWI Prolog based
on delimited control operators for Prolog recently introduced by \cite{SchrijversDemoenDesouter2013}.
In comparison with a previous implementation of tabling for SWI Prolog using delimited
control \cite{DesouterVan-DoorenSchrijvers2015}, this approach, based on 
the functional memoising parser combinators of \cite{Johnson1995}, stays closer to
the declarative core of Prolog, requires less code, and is able deliver solutions
from systems of tabled predicates incrementally (as opposed to finding all solutions
before delivering any to the rest of the program).

A collection of benchmarks shows that a small number of carefully targeted optimisations
yields performance similar to the highly optimised version of Desouter et al's system
currently included in SWI Prolog.
}
